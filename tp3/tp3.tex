\documentclass[14pt]{extarticle}

\DeclareMathSizes{14}{14}{15}{12}

\renewcommand{\normalsize}{\fontsize{14}{16}\selectfont}


% Language setting
% Replace `english' with e.g. `spanish' to change the document language
\usepackage[english]{babel}
\usepackage{graphicx}
\graphicspath{ {./images/} }
% Set page size and margins
% Replace `letterpaper' with `a4paper' for UK/EU standard size

\usepackage[letterpaper,top=2cm,bottom=2cm,left=3cm,right=3cm,marginparwidth=1.75cm]{geometry}

% Useful packages
\usepackage{amsmath}
\usepackage{graphicx}
\usepackage[colorlinks=true, allcolors=blue]{hyperref}

\title{TP3 Automatique - Sami BOUFASSA}

\begin{document}
\maketitle


Question 1 :

\begin{figure}[tbh]
    \vspace{0.1cm}
        \centering
        \includegraphics[width=\columnwidth]{"images/q1.jpg"}
    
    \end{figure}



Question 2 :

\begin{itemize}
    \item Le vecteur d'etat X contient toutes les variables necessaires pour decrire le mouvement du Lunar Lander autrement dit les positions (x,y) et les vitesses ($\dot x$, $\dot y$). 
    \item La derivee du vecteur d'etat est bien une combinaison lineaire des etats et des entrees $a_x$ et $a_y$ des moteurs du Lunar Lander.
    \item La sortie Y est une combinaison lineaire des etats du vecteur d'etat (et des entrees). 
    \item Toutes les lignes de la derivee du vecteur d'etat sont lineairement independantes. 
\end{itemize}


Questions 3 et 4 :
\begin{figure}[tbh]
    \vspace{0.1cm}
        \centering
        \includegraphics[width=\columnwidth]{"images/q3_q4.jpg"}
    
    \end{figure}





\break 
Question 5 : 

\begin{figure}[tbh]
    \vspace{0.1cm}
        \centering
        \includegraphics[width=\columnwidth]{"images/q5.jpg"}
    
    \end{figure}

\vspace{0.5cm}

Question 6 :

    Les équations gouvernant le mouvement en x ne dépendent que des variables associées à x (position et vitesse en x), tandis que celles en y ne sont fonction que des variables liées à y (position et vitesse en y).

\break
Questions 7 et 8 :
\newline 

\begin{figure} [tbh]
\vspace{0.1cm}
\centering
\includegraphics[width=\columnwidth]{"images/q7_q8.jpg"}
        
\end{figure}

\break 
\begin{figure} [tbh]
    \vspace{0.1cm}
        \centering
        \includegraphics[width=\columnwidth]{"images/q11_q12.jpg"}
    
    \end{figure}


\end{document}
	
